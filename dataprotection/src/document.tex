\documentclass{article}
\usepackage{todonotes}
\linespread{1.3}

\begin{document}


\section{Introduction}
\todo{$1/2$ page}

\section{Commercial, Free and Open Source Software}
In this section, we will give an overview over the different methodologies
in software development with respect to sharing their source code.
\todo{intro}
\subsection{Source Code}
In 1936, Alan Turing introduced the ``a-machine''
(now called the \emph{Turing-Machine}):
A hypothetical device that reads and manipulates an infinitely long
\emph{tape}, which has symbols written on it.
Turing showed that, given the correct configuration of such a machine,
it would be able to execute any desired algorithm on the input
symbols and therefore calculate any \emph{computable} number.
Furthermore, Turing showed that a special version of such a machine,
a \emph{universal} Turing-Machine, could emulate any of the above
machines by encoding the configuration of the emulated machine
(the desired algorithm) onto the input tape in front of the actual
input data on which the algorithm would operate. 
Whereas the Turing-Machine has a strong presence in theoretical
computer science, where it helps researchers understand the limits of
mechanical computation, it has also been a foundation of practical
computing. Focus shifted from electronic hardware that was
specifically built for one purpose, say an electronic calculator
that adds two numbers, to hardware that could read and execute any
desired instruction, in this case ``add'', before also reading the
input data for this instruction.

This was the origin of computers as we know them today.
The \emph{Central Processing Unit}, or CPU, is an attempt to realize the
concept of a Turing-Machine in the real world and its input tape is
realized by the \emph{Random Access Memory}, or RAM, which is present
in any computer. The CPU will fetch instructions as well as the data to
execute these instructions on from the RAM and will write the results
back to the RAM. The sequence of instructions executed by the CPU are
commonly called a \emph{computer program}. The instructions themselves
are very basic (e.g. ``add'', ``invert'', ``AND''), but by sequencing
them correctly, one can encode any desired algorithm, from simple
addition to programs like Microsoft Word or Google Earth.

Given the degree of complexity of modern software, it is clear that
it would be practically impossible to generate something like Google
Earth from these basic instructions within reasonable time. For this,
there are programs called \emph{compilers}. By using a compiler, a
programmer is able to write instructions in a \emph{high level language},
such as C or Java, where he has much more powerful tools at his
disposal. The compiler will then translate this high level language
into the basic instructions from above and thus make it readable to
the CPU. The generated basic instruction sequence is commonly referred
to as the \emph{machine code} or \emph{executable} of a program,
whereas the high level
text is called the \emph{source code}. Since the process of compiling
is not an \emph{injective} function, it is generally very hard
(even theoretically impossible) to obtain the original source code
from the machine code. This provides a degree of secrecy, especially
given the vast complexity of machine code, and a series of issues
have emerged about how to handle this secrecy. In the following, we
will present the three main approaches to these issues. 
\subsection{Commercial Software}
Commercial software is the classic form of software. It is usually
developed
within a company in response to a customer's request, as a new product
to be launched or as an in-house tool to enhance productivity.
Since the company's employees earn some sort of salary, newly
developed software represents a substantial amount of work that
has been put into developing it and naturally, the company would like
to be reimbursed for that by whomever uses that software. The most
logical way to do this is to charge users a fee via either a 
\emph{purchase} or a periodic \emph{subscription fee}.

Within such an environment, the customer is often given just the
executable to run the program. The company will usually keep the
source code as a secret for various reasons:
\begin{itemize}
  \item Distributing the source code would make it much easier to
illegally share the software.
\item Security holes in the code may be found easily by hackers.
\item Secret algorithms may be openly visible.
\end{itemize}
In general, commercial software producers will argue that
distributing their source code will allow people to take advantage
of the work the company put into development without having to
reimburse the company for it.
\subsection{GNU and the Free Software Foundation}
As the tale goes, one day, \emph{Richard Stallman} found a bug
in the driver software for a printer at his university. With
the intention of fixing the but, he asked the vendor for the
driver's source code, but the vendor didn't comply. Stallman then
decided to start \emph{GNU}:
\begin{quote}
Starting this Thanksgiving I am going to write a complete
Unix-compatible software system called GNU (for Gnu's Not Unix),
and give it away free to everyone who can use it.
Contributions of time, money, programs and equipment
are greatly needed. \todo{citation}
\end{quote} 
He later went on to start the \emph{Free Software Foundation}
dedicated to promote the spreading of free software. \emph{Free},
in this context, has the meaning of \emph{free speech} rather than
\emph{free beer}.

The Free Software Foundation strongly disagrees with the views held
by commercial software vendors and argues that all software should
be free software. 
In order for a program to be classified as free
software, the foundation states:
\begin{quote}
A program is free software if the program's users have
the four essential freedoms:
\begin{itemize}
  \item The freedom to run the program, for any purpose.
  \item The freedom to study how the program works,
and change it so it does your computing as you wish.
Access to the source code is a precondition for this.
  \item The freedom to redistribute copies so you can help
your neighbor.
\item The freedom to distribute copies of your modified versions to
others. By doing this you can give the whole community a chance
to benefit from your changes. Access to the source code is a
precondition for this.
\end{itemize}
\todo{cite}
\end{quote}
To promote this approach, the foundation has been active on the
technical as well as on the legal side and one of its greater
accomplishments in the latter is the \emph{GNU General Public
License}, further discussed in section \todo{cite}.
\subsection{The Open Source Initiative}
As the Free Software movement began to gain momentum, developers
started to notice something, which might have been unexpected for
many of them: Free software proved to be very good qualitatively.
The philosophy and legal framework made it possible for anyone using
the software to improve it and share his or her improvement
with the whole community, resulting in faster and better improvements
than most companies could generate. Given its quality, it is not
astonishing that
today, a large part of webservers, business infrastructures and
smartphones are running a Linux kernel.

Seeing this, many developers started to sense the possibility to
create high quality software through such an approach, but there
were a lot of concerns about the term \emph{Free Software}. In
particular, it is hard to get businesses involved in a process
dubbed \emph{free} and accompanied by the whole philosophical
argument around it. In response to that, the term
\emph{Open Source Software} was coined and the
\emph{Open Source Initiative} was founded. The goal was to shift
the focus away from the ethical arguments about the users' freedom
and towards the 
practical benefits and the higher quality of resulting software
when using an open process.
\begin{quote}
The fundamental difference between the two movements is in their
values, their ways of looking at the world. For the Open Source
movement, the issue of whether software should be open source
is a practical question, not an ethical one. As one person put it,
``Open source is a development methodology; free software is a social
movement.'' For the Open Source movement, non-free software is a
suboptimal solution. For the Free Software movement, non-free
software is a social problem and free software is the solution.
\todo{cite}
\end{quote}
In legal terms, the Open Source
definition does not differ much from that of Free Software and there
are few instances, where they are incompatible. We will see
such instances in section \todo{ref}.
\todo{end}
\section{Comparison of Software Licences}
\todo{$1$ page}
\todo{intro}
\todo{warranty}
\todo{copyright}
\subsection{The GPL}
\todo{cite}
Out of the philosophy around free software, the 
\emph{GNU General Public License} has been
created in order to protect the fundamental freedoms of the users.
Under the GPL, users are granted permission to 
\quote{use, share, study and modify} the program if they wish.
A central theme is the use of \emph{Copyleft} in order to ensure
that the original author's work will sustain these rights in
derived works:
\begin{quote}
Copyleft says that anyone who redistributes the software,
with or without changes, must pass along the freedom to further
copy and change it. Copyleft guarantees that every user has freedom.
\todo{cite}
\end{quote}
This, in essence, means that if a programmer uses source code
licensed under the GPL in his or her program, then that program
must also be licensed under the GPL.
\subsection{The LGPL}
\todo{cite}
\subsection{The MIT and BSD licenses}
\todo{cite}
\todo{end}

\section{Relation to data protection}
\todo{$1$ page}
\todo{intro}
\todo{end}

\section{Ethical Issues}
\todo{$3/2$ page}
\todo{intro}
\todo{end}

\section{Conclusions}
\todo{$1/2$ page}

\end{document}
